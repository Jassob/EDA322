\documentclass{article}

\title{EDA322 Digital Design -- Exam kit \\
  \small Revision: 1 (2018-06) -- Taube \& Jassob
  %% \normal Revision: X (date) -- Authors
}

\usepackage[includeheadfoot, top=1.0cm, bottom=2cm]{geometry}
\usepackage{hyperref, graphicx, parskip, lastpage}

\begin{document}
\maketitle
\tableofcontents

\newpage

\section{Introduction}
This is an exam kit for the course Digital Design given at Chalmers
University of Technology to second-year student on the Computer
Science and Engineering program.

The course is given by the Computer Engineering division at the
Computer Science and Engineering department.

The course is at the time of writing (2018) given in English and
therefore this exam kit is also written in English.

\subsection{Motivation}
Digital Design is a course that covers {\it a lot} and since there
usually are not very good answers to old exams it can be a difficult
course to retake if one does not attend every lecture.

This exam kit attempts to make it easier to study the course without
the aid of the lectures by including more thorough explanations and,
when needed, visualizing such explanations with step-by-step pictures.

\subsection{Structure}

This exam kit is divided into three sections:
\begin{itemize}
\item Introduction (which you are currently reading);
\item Theory \\
  contains a summary of the important parts of the course's theory
\item Quiz \\
  contains answers to quizzes given in lectures, good for quickly
  assessing what you might need to study more.
\end{itemize}

\newpage

\section{Theory}

\subsection{Configurable and non-configurable hardware}
There are different types of circuits and they come with their own set
of advantages and disadvantages. In this course we make the
distinction between reconfigurable hardware and non-reconfigurable
hardware. The two main types of circuits discussed in the course is
ASICs and FPGAs.

\subsubsection{ASIC (Application Specific Integrated Circuit)}
A non-reconfigurable circuit. The hardware of an ASIC can not be
changed after production. The static, specialized nature of this kind
of chips makes them energy-efficient and high-performing, but also
expensive to produce due to its high NRE\footnote{Non-recurring
  engineering, the one-time cost to research, design, develop and test
  a product} cost. Viable if many chips of the same kind and a
specific purpose is required (e.g. graphic card).

\subsubsection{FPGA (Field Programmable Gate Array)}
A specific type of reconfigurable circuit that contains a lot of
different circuit elements (for instance logic blocks, interconnects,
RAM memory blocks, multipliers etc) and is configurable by a
bitstream. This bitstream specifies how the circuit elements should
behave and the way they are connected to each other. Therefore, by
changing the configuration bitstream of an FPGA one can change the
functionality of it, or reconfigure it.

Compared to implementing circuit logic in ASICs, an FPGA design would
probably result in higher power consumption and often lower
performance, but since the chip is reconfigurable flaws in the design
can be fixed by updating the design and loading it to the FPGA,
resulting in a lower NRE cost (both in materials and time due to
quicker ``verify-and-fix'' iterations). A common use of FPGAs is
therefore to prototype a circuit design on an FPGA and then create
ASICs when the design is final.

FPGAs are also good choice if the volume of circuits is not big enough
to motivate the high cost of producing ASICs or if the flexibility of
reconfiguration is valuable (for instance to fix hardware bugs in a
system, or just re-use if a use case is not needed anymore).

For more information of FPGAs and their applications, we refer to the
wikipedia
\href{https://en.wikipedia.org/wiki/Field-programmable_gate_array#Applications}{article}
or section \ref{sec:rec-hw} which describes reconfigurable hardware in
more detail.
\section{Quiz}

\end{document}
