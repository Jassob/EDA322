\documentclass{article}

\title{EDA322 Digital Design -- Exam kit \\
  \small Revision: 1 (2018-06) -- Taube \& Jassob
  %% \normal Revision: X (date) -- Authors
}

\usepackage[includeheadfoot, top=1.0cm, bottom=2cm]{geometry}
\usepackage{hyperref, graphicx, parskip, lastpage}

\begin{document}
\maketitle
\tableofcontents

\newpage

\section{Introduction}
This is an exam kit for the course Digital Design given at Chalmers
University of Technology to second-year student on the Computer
Science and Engineering program.

The course is given by the Computer Engineering division at the
Computer Science and Engineering department.

The course is at the time of writing (2018) given in English and
therefore this exam kit is also written in English.

\subsection{Motivation}
Digital Design is a course that covers {\it a lot} and since there
usually are not very good answers to old exams it can be a difficult
course to retake if one does not attend every lecture.

This exam kit attempts to make it easier to study the course without
the aid of the lectures by including more thorough explanations and,
when needed, visualizing such explanations with step-by-step pictures.

\subsection{Structure}

This exam kit is divided into three sections:
\begin{itemize}
\item Introduction (which you are currently reading);
\item Theory \\
  contains a summary of the important parts of the course's theory
\item Quiz \\
  contains answers to quizzes given in lectures, good for quickly
  assessing what you might need to study more.
\end{itemize}

\newpage

\section{Theory}
\subsection{Circuits}
\subsubsection{ASIC (Application Specific Integrated Circuit)}
A non-reconfigurable circuit. The hardware of an ASIC can not be
changed after production. The static, specialized nature of this kind
of chips makes them energy-efficient and high-performing, but also
expensive to produce due to its high NRE\footnote{Non-recurring
  engineering, the one-time cost to research, design, develop and test
  a product} cost. Viable if many chips of the same kind and a
specific purpose is required (e.g. graphic card).

\subsubsection{FPGA (Field Programmable Gate Array)}
A configurable circuit. The hardware of an FPGA supports different
designs. A design translates to a specific bitstream and thus by
changing the configuration bitstream of a FPGA it is possible to
reconfigure the design.

Compared to implementing circuit logic in ASICs, implementing it using
a FPGA design would probably result in a lower NRE cost, but a higher
power consumption and often lower performance. FPGAs are a good choice
if the volume of circuits is not big enough to motivate the high cost
of producing ASICs or if the flexibility of reconfiguration is
valuable (for instance to fix hardware bugs in a system, or just
re-use if a use case is not needed anymore).

A common use of FPGAs is to prototype a component design and test it
on a FPGA and then create ASICs when the design is final.

For more information of FPGAs and their applications, we refer to the
wikipedia
\href{https://en.wikipedia.org/wiki/Field-programmable_gate_array#Applications}{article}
or section \ref{sec:rec-hw} which describes reconfigurable hardware in
more detail.
\section{Quiz}

\end{document}
